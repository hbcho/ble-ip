\section{Introduction} \label{intro}
Indoor positioning(IP) is moving towards becoming ubiquitous and fulfilling the goal of Internet of Things(IoT) and Artificial Intelligence(AI) of Location Aware services\improvement{[cite here]}.  The explosion of usage of smartphones in the last decade has also enabled an exponential growth of this field. IP mainly leverages the sensor hub in smartphones like accelerometer, gyroscope, barometer, proximity, thermometer, pedometer, magnetic sensor, etc and wireless technologies like bluetooth and WiFi for accurately solving this problem. Signals of opportunities like magnetic field, pressure, light and sound intensity, Global Navigation Satellite System [Torres-Sospedra et al, A realistic evaluation of indoor positioning systems based on Wi-Fi fingerprinting, 2017], mobile network \cite{99} and frequency modulation(FM)\cite{98} signals have also been used which don't incur any additional infrastructure costs.  IP also gets impetus from interrelated fields like geo-location, augmented reality, natural user interface, gaming, man-machine interaction, and 3D mapping.

\vspace{3mm}
The need for indoor positioning arose because the traditional Global Positioning System(GPS) fares badly indoors due to signals attenuation and scattering caused by roofs and walls. This leads to higher uncertainty in estimation of location which sometimes spans across multiple rooms sabotaging the whole positioning problem. \improvement{put it to some other good place} Usage of indoor positioning(IP) can also be reasoned by following reasons \change{change the adjective here, not quite intuitive} condition, \unsure{right word?} convenience and commercialization. Condition in the sense people tend to stay indoors 90\% of the time according to Environment Protection Agency(EPA) \cite{geospatial}. Convenience as it can be used for locating rooms and tracking assets. Commercialization, in case of malls, groceries stores, can be used to send out discount coupons, personalized recommender engines. The positioning data could also be used to optimize the resource flow in hospitals or warehouse to minimize the operation cost\cite{max}.

\vspace{3mm}
Formally, the estimation of location of mobile unit(MU) using wireless technologies is called location sensing, geo-location, position location or radio-location\cite{liu}. The conventional GPS suffer with problems like building occlusion, multipath effect \cite{fang} and signal attenuation \cite{geospatial} \improvement{cite classic text book} indoors, and hence many other technologies have been advocated. \improvement{improve it} Even though the GPS technology can be improved using the GPS repeaters, the solution is not scalable and would require costly infrastructure. The new technologies for positioning include Assisted GPS, \thiswillnotshow{GPS + cellular mobile tower signal} Bluetooth Low Energy(BLE), MEMS sensors, LED, WiFi, Magnetic anomalies \unsure{is it fine if I use abbreviations directly} or methods used in conjunction with each other \cite{geospatial}. 

IP has been solved using the triangulation techniques like lateration with  time of arrival(TOA), time difference of arrival(TDOA), angle of arrival(AOA), received signal strength(RSS) or fingerprinting based scene analysis methods like decision based probabilistic method, k-Nearest Neighbour(k-NN), artificial neural networks(ANN) or proximity based methods\cite{liu} and Gaussian processes with latent variable models\cite{ferris}. For the rest of the article, we use beacons and nodes interchangeably. 
 \\

\info[inline]{ 
 1. AOA needs at least two known sources. 2. i-locate, open geo-data. \url{http://www.i-locate.eu/pilot-sites/}. 3. Before the advent of  bluetooth low energy(BLE) beacons, the indoor positioning was ....}

\subsection{Setup}
The BLE beacons are part intelligent luminaires are densely packed, the nodes can read the signal strengths from neighboring nodes. Generally, a location estimation task involves two hardware devices, one transmitting the signal and other receiving it\cite{liu}. Here, the transmitting device is bluetooth low energy chip(or WiFi router?) installed in Active Aheads. The other sensor data (there is no infrastructure for getting these data as of now),  follow-strength values is planned to be downloaded once and updated biweekly or monthly. The Active Aheads are part of a mesh network.